\documentclass[11pt]{article}
\usepackage[margin=0.8in]{geometry}
\usepackage{bm}
\usepackage{amsfonts}
\usepackage{amsthm}
\usepackage{amssymb}
\usepackage{amsmath}
\usepackage{xcolor}

\title{\textbf{Option Valuation with the Fast Fourier Transform Summary}}
\author{Andrew Wu}
\date{March 2021}

\usepackage{titlesec}
\titleformat{\section}{\Large\sffamily\bfseries}{\thesection}{0.5em}{}[]

\newcommand{\E}{\mathbb{E}}
\newcommand{\R}{\mathbb{R}}
\newcommand{\Q}{\mathcal{Q}}

\begin{document}
	\maketitle
	\section{Motivation}
		\subsection{Deficiencies of Black-Scholes}
		\begin{itemize}
			\item GBM assumption in the standard Black-Scholes model is insufficient for the market as returns tend to be more peaked with longer tails than the normal distribution (leptokurtic) (and volatility smile). 
			\item Want to use more complicated underlying processes that can include jumps and account for leptokurtic returns.
			\item Complicated underlying processes may have complicated densities (with no closed form), so the standard way of pricing a call via integration is difficult:
			\[	C_{T}(k) = e^{-rT}\int_{k}^{\infty} (e^{s} - e^{k}) q_{T}(s)ds
					\]
			where $q_{T}(s)$ is the density of the terminal log-stock price.
			\end{itemize}
		\subsection{Fourier methods}
		
		\begin{itemize}
			\item Characteristic functions usually more well behaved than the density and may have closed form. 
			\item Goal: Express Fourier transform of call price (or components of it) in terms of the characteristic function of terminal log-stock \( \phi_{T} \), then use the inverse Fourier transform to recover the call price.
			\item For example: Scott 1997, Bakshi/Madan 1999 give the formulas
			\begin{align*}
				\text{Delta}:=\Pi_1 &= \frac{1}{2} + \frac{1}{\pi}\int_{0}^{\infty}\Re\left(\frac{e^{-iu\ln K}\phi_{T}(u - i)}{iu\phi_{T}(-i)}\right)du.\\
				\mathbb{P}\left(S_T > K\right) := \Pi_{2} &=   \frac{1}{2} + \frac{1}{\pi}\int_{0}^{\infty} \text{Re} \left(\frac{e^{-iu\ln K }\phi_{T}(u)}{iu}\right)du.\\
				&\text{Price} = S_0 \Pi_{1} - Ke^{-rT} \Pi_{2}.
				\end{align*}
		
			\item Directly computing the integrals above is slow and doesn't take advantage of FFT algorithms. 
			\end{itemize}
		
		\section{Carr and Madan's Fourier method}
			\subsection{Notation and definitions}
			Let:
			\begin{itemize}
				\item	\( s_{T} = \ln(S_T) \), \( k = \ln(K) \) be the log terminal stock and  log-strike prices respectively.
				\item	\( \phi_{T} \) be the characteristic function of the log-stock.
				\item \( C_{T}(k) \) be the price of a call option with maturity \( T \) and log-strike \( k \).
			\end{itemize}
			\subsection{Modified call}
			\begin{itemize}
				\item Call price \( C_{T}(k) \) is not integrable, multiply by factor \( e^{\alpha k} \) where \( \alpha >0 \) is a parameter to be specified. Define the modified call option \( c_{T}(k) \):
				\[		c_{T}(k) = e^{\alpha k}C_{T}(k).
						\]
				\item Express the Fourier transform \( \psi_{T}(v) \) of modified call in terms of characteristic function.
				\[		\psi_{T}(v) = \frac{e^{-rT}\phi_{T}(v - (\alpha + 1)i)}{\alpha^{2} + \alpha - v^2 + i(2\alpha + 1)v}
						\]
				Notice that if \( \alpha = 0 \) we would have a singularity at \( v = 0 \). Also, DFT needs to evaluate \( \psi_{T} \) at \( v = 0 \).
				\item The call price can be recovered by taking the inverse Fourier Transform and dividing by the modification factor and some symmetry arguments since \( C_{T}(k) \) is real:
				% JUSTIFY THE SYMMETRY ARGUMENTS?
					\[	C_{T}(k) = \frac{e^{-\alpha k}}{\pi}\int_{0}^{\infty}e^{-ivk}\psi_{T}(v)dv.
							\]	
				\end{itemize}
		\subsection{Fast Fourier Transform}	
		\begin{itemize}
			\item	Definition of DFT: The Discrete Fourier Transform (DFT) transforms a sequence of complex numbers \( \{x_j\}_{j = 0,1,...N-1} \) into the sequence \(\{X_{k}\}_{k=0,1,...N-1}\), where 
			\begin{align}
				X_{k} = \sum_{j = 0}^{N - 1}e^{-2\pi i k j/ N}x_{j}.\label{eq:1}
			\end{align} 
			\item Suppose we use \( N \) partition points $v_j = \eta j$, $j=0, 1, \ldots N$, where $\eta$ is the spacing size and \( a = N\eta \) is the upper integration limit (need to truncate the integral as infinite upper bound):
			\item Then we have the following DFT approximation
			\[	C_{T}(k)\approx \frac{e^{-\alpha k}}{\pi}\int_{0}^{a} e^{-ikv}\psi_{T}(v)dv \approx \sum_{j=0}^{N-1}e^{-ikv_j}\psi_{T}(v_j)\eta.
					\]
			\item Now define a series of log-strikes in order to use FFT. Let 
			\[	k_u = -\frac{N\lambda}{2} + u\lambda
					\]
			where $u=0,1,2\ldots N - 1$, yielding log-strikes uniformly spaced in \( [-\frac{N\lambda}{2}, \frac{N\lambda}{2}) \).
			\item To use DFT, set 
			\[	\lambda = \frac{2\pi }{N\eta}
					\]
			which unfortunately yields a tradeoff between integral accuracy (smaller \( \eta \)) and increasing strike spacing (larger \( \eta \)).
			\item Use Simpsons rule weights to achieve more accurate integral estimate:
			\[	C(k_u) \approx \frac{e^{-\alpha k_u}}{\pi}\sum_{j=0}^{N-1} e^{-2\pi iju/N}e^{\pi j} \psi_{T}(v_j)\cdot \frac{\eta}{3}\left[3 + (-1)^{j+1} - \delta_{j}\right].
					\]
  		\end{itemize}
		
		\section{Application and Implementation}
			\subsection{Variance gamma process}
				\begin{itemize}
					\item	The stock price is assumed to be driven by a Variance-Gamma process, a 1-D pure-jump Markov process:
					
					\[	S_t = S_0\exp\{	
										(r + \omega) t + X_{t}(\sigma, \theta, \nu)
											\},
							\] 
					\item	By default, we set 
					\[	\omega = \frac{1}{\nu} \ln(1 - \theta \nu - \frac{1}{2}\sigma^2\nu),
								\]
					so that the mean rate of return on the stock is \( r \).
					\item \( X_t \) is calculated by evaluating an arithmetic Brownian motion with drift \( \theta \) and volatility \( \sigma \) at a random time \( \Gamma(t; 1,\nu) \) which is a Gamma process, a pure jump process with independent increments that follow a gamma distribution.
					\[	X_t(\theta, \sigma, \nu) = \theta \Gamma(t;1, \nu) + \sigma W(\Gamma(t; 1, \nu)).
							\]
					\item Variance Gamma is supported in general equilibrium model and addresses volatility smile/no jumps in the traditional GBM assumption of the Black-Scholes Model.
					\item Variance Gamma also has complicated density (not closed form but analytic due to Bessel function), however characteristic function is relatively simple, so it is ideal for testing this approach.
					\item Characteristic function of terminal log-stock \( s_T \) is:
					\[	\phi_{T}(u) = (1 - i\theta\nu u + \frac{1}{2}\nu\sigma^2 u^2)^{-T/\nu}
								 		\exp\left\{
								 			(
								 				\ln(S_0) + (r + \omega)T
								 			)
								 			iu
											\right\}
								\]
				\end{itemize}
			\subsection{Code}
				\begin{itemize}
					\item Used parameter combination 4 in Carr Madan paper:
					\[	r = 0.05, S_0 = 100, \sigma = 0.25, \nu = 2, \theta = -0.1.
							\]
					\item For GBM, Fast Fourier Transform often gave an error of between 1 - 2.
					\item Fast Fourier Transform gave a significant time saving but accuracy was poor across all Fourier methods for shorter maturities, making comparison difficult.	 
				\end{itemize}
		\section{Conclusion}
			\begin{itemize}
				\item	FFT has potential to be useful, perhaps some better quadrature rules can be used than Simpson's rule to increase accuracy. 
				\item	More robust than some of the other Fourier methods, which resulted in large errors due to the pole at \( 0 \) in the integral.
				\item However, computes unnecessary stock prices that may be far out or in the money (although still faster) due to the requirement for no. of strikes = no. partition points in quadrature.
				\item Note: Call option when VG used as underlying has closed form in terms of confluent hypergeometric functions but I was unable to implement this in Python without errors in my integrals. Matsuda (2004) experienced the same issues.
			\end{itemize}
\end{document}